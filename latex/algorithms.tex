\section{Algorithms}
\subsection{Theory}
\subsubsection{PID controller}
% Translated from Garipov, Digital Control Systems 1, page 33.
The PID controller is an enchancement of the PI controller.
An aditional compoennt is assed to the PI sum.
It depends on the speed of change of the error but is expected to be zero at steady-state.
Here is how PID control looks like in the time domain:
\begin{equation}
    u(t) = K_p e(t) + K_p \frac{1}{T_i s} \int_0^{t_1} e(t) dt + K_p T_d \frac{d e(t)}{dt} = u_p(t) + u_i(t) + u_d(t)
\end{equation}
where $T_d$ is caleld derivative time constant.
\par
In state space the PID control law looks like this:
\begin{equation}
    u(s) = K_p e(s) + K_p \frac{1}{T_i s} + K_p T_d s e(s) = u_p(s) + u_i(s) + u_d(s)
\end{equation}


\subsection{PID implementation}
Used is a discret pid algorithm with parallel structure and integral windup.
The computer implementation expects proportional integral and derivative gains, as opposed to time constants.
Consequencly, the responsibility for translating from $T_i, T_d$ to $K_i, K_d$ falls to the client software, written by the author.

\subsection{Units of measure}
Because various physical and computational resources are performed in the system, unit conversions need to be carefully considered.
\par
We are constrained by the PID algorithm software to use 'int16\_t' values for input, output and coefficients.
Furthermore, coefficients are represented in 9s6 format i.e. $128 == 1.0$.
\par
Secondly, the temperature sensor interface library returns 'decicelsius\_t'.
This is an integral value of measured degrees celsius, multiplied by 10.
The accuracy of the thermometer is $0.5\si{\celsius}$
\par
Last constraint is the input to the triac control.
In an attempt to increase the resolution of the triac control algorithm as far as possible, the author has used the whole range of 'uint16\_t' values.
In other words, 0 corresponds to 0\% or 0\si{\volt} at the heater, and $2^16 \equiv 65536$ corresponds to 100\% or 230\si{\volt} at the heater.
\par
Apparently, the PID algorithm needs to read 'decicelsius\_t' and output 'uint16\_t', which is impossible.
One of the two units needs to be selected for the algorithm to work with, and the other - converted to/from.
The author has selected 'decicelsius\_t' as the more human-readable format.
Furthermore, no aditional precision is lost by using this shriked type.

\subsection{Nonlinearity}
The heating circuit is expected to exhibit good linearity.
However, the cooling process is both not under the control of the regulator, and extremely non-linear.
The author expects cooling to conform to the Stefan-Bolzman law of radiated thermal energy in free space:
$$ L = \frac{\sigma}{\pi}T^4 $$
If this is true, estimating a transfer function of the whole system is pointless.
The resulting linear system will be close to reality only for a very limited range of setpoints.

\subsection{Ziegler-Nichols sustained oscillations}
Initially, the Ziegler-Nichols algorithm for obtaining sustained oscillations and thus critical gain $K_u$ and critical period $T_u$ was attempted.
It was observed that if any oscillations are observed at all, their amplitude is lower than the absolute error in the system.
\includegraphics[width=0.85\textwidth]{../images/exp_gain100}~

\subsection{Astrom-Hagglund sustained oscillations}
Secondly, replacing the PID controller with a relay was attempted.
The relay acts as a sign function.
It was observed that again no sustained oscillations are exhibited by the system.
\includegraphics[width=0.85\textwidth]{../images/exp_relay}~

\subsection{Sampling rate correction}
Following advice in the literature, the author increased the control algorithm period to roughly $1/10$'th of the dominant open loop system time constant, namely to 60 seconds.
Finally sustained oscillations were observed.
\\
\includegraphics[width=0.85\textwidth]{../images/exp_relay_slow}~


