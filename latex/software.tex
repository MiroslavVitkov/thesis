\section{Software}
\subsection{Theory}
\subsubsection(Radix tree)
The radix tree is a data structure, suitable for storing and searching sets of strings.
Each node, that is an only child, is merged with its parent.
The operations, that can be performed on a radix tree, are as follows:
\begin{itemize}
\item{insertion}
\item{deletion}
\item{searching}
\item{find all strings with common prefix}
\item{find predecessor}
\item{find successor}
\end{itemize}
All those operations exhibit complexity of O(k), where k is the length of the string.
\par
On the other hand, radix trees can only be applied to strings, or to sets wich are directly mappable to strings.
They also need serializable types to work on.

\subsection{Goals}
The device is a typical realtime embedded system.
As such, mainframe programming tecniques are not applicable and embedded systems approach is required.
Consequently, the following requirements are layed out to the program:
\begin{itemize}
\item{Conveninet programming - during development, reprogramming the device shall be quick and easy, preferrably consisting of a single action.}
\item{Convenient communication - both debug data and process information shall be easily accessible during device operation.}
\item{Error tolerance - the device shall not freeze up upon an error, but should instead make best effort to keep the control loop active.}
\item{Determinism - the programmer needs to have full control over what is executed when.}
\item{Scriptability - the device shall be able to get reconfigured without a human at the other end id est a PC program shall be provided, which reconfigures the device in arbitrary ways.}
\end{itemize}
In the oppinion of the author, all of these requirements have been fully fufilled with the current version of the software.
The written software is free and open-source in it's entirety - it is protected under the MIT license.

\subsection{Language selection}
The selected processor can be programmed in avr-assembler, C or C++.
Additionally, compilers exist for other less widespread languages, such as ADA.
%TODO: quote
The author has selected the C language for it's simplicity, widespread use and compiler support for all of it's features.

\subsection{Bootloader}
\subsubsection{Motivation}
Several approaches were evaluated to reprogram the chip.
\par
As the chip resides in a socket, instead of being soldered to the Main board, it could be removed from it's socket and inserted into a programmer.
This approach is useful, but was discarded as extremely time-consuming.
\par
The next possible approach is In-system programming.
%TODO: quote something
This constitutes soldering a header onto the Main board, and connecting it to the SPI bus of the microcontroller.
This approach was also rejected for hte following reasons.
The author was relucant to pay both the price in real estate on the PCB and the aditional complexity of wirein the header to the microcontroller.
Not to mention using up some pins.
Last but not least, this approach would significantly complicate the RESET pin nooise protection circuit.
\par
The most complex, but in the same time best performing, approach was selected.
Utilizing the fact that serial communication to a PC is needed for other reasons, the author decided to transfer new programs over the same bus.
This way the device can be reprogrammed purely bu software, possibly even from a remote location.

\subsubsection{Previous work}
Many bootloader programs exist for the selected chip.
%TODO: quote 3-4
Unfortunately, all of them are either:
\begin{itemize}
\item{bloated with unnececery functionality, thus large in size,}
\item{do not work out of the box and thus require manipulation of the source code to work or}
\item{are licensed under un incompatible license than the veri permissive MIT license.}
\end{itemize}

\subsubsection{Implementation}
The bootloader was implemented in standard-conforming C language, and written by the author in it's entirety.
It is nearly impossible to create smaller in size bootloader for this device.
%TODO: quote avr-freaks
The program code is simple and straighforward, lacking complicated preprocessor directives.
Consequencly, it is in the oppinion of the author, easy to read and modify by anyone.
The code is of cource publically visible and licensed in a way, that any person can freely and legally modify and redistribute it.
\par
The bootloader instructions reside at the high end of the flash memory, while the application resides in the low end.
%TODO: quote the datasheet
Consequencly, the application never knows that a bootloader program was installed.
\par
The finished product was called 'megaboot' - a bootloader for atmega devices.
The source code in its entirety is provided as appendix A1.

\subsubsection{Communication protocol}
The XMODEM protocol has been selected.
%TODO: quote
This protocol is not used in modern high-bandwidth, high-complexity networking.
To the contrary. this protocol was popular when computers posessed resources comparable to the selected microcontroller.
It is a simple protocol with very little overhead (channel efficiency is 97\%).
Furthermore, it is easy to implement in few program instruction, using up small amount of flash memory.

\subsubsection{Error checking}
Erorr checking is provided throughtout the program:
\begin{itemize}
\item{CRC sums onto each received XMODEM packet}
\item{"magic character" starts each packet}
\item{addressing wrong (inexistient) flash pages is protected against}
\item{receiving packets out of order is also checked agains}
\end{itemize}
All the error checking code is conditionally included via the only preprocessor definition.
This serves twin purpose.
Not only does it clearly indicate error checking code apart from the actual program logic, but also provides the option to remove all error checking, should a person attempt to reduce the program size to the next smaller option.

\subsubsection{Use}
The build system passes the symbol BOOTLOAD to the application program.
This symbol holds the address of the first instruction of the bootloader.
The application is then free to jump to that address whenever required.
In an assembler program, this would have been performed via a RJMP instruction.
On the other hand, higher level languages cast the pointer to a function call address (optionally with the \_Noreturn attribute) and call it.
\par
After the bootloader has been invoked, it expects program data over the communication channel.
Fortunately, many applications are available, which support the XMODEM protocol.
One example is the popular and free serial terminal 'minicom'.

\subsection{External components}
\subsubsection{Onewire library}
A C library, written by Peter Dannegger, Martin Thomas and others, is being used for communication with the thermometer.

\subsubsection{PID implementation}
A professional implementation of a parallel pid with integral saturation, implemented by Atmel employees, is used.
% TODO: quote the app note

\subsection{Architecture}
\subsubsection{Theory}
Numerous program architectures have been published throughout embedded programming books.
%TODO: quote
However, they all fall in one of the following three groups:
\begin{itemize}
\item{Continuous execution - suitable for very simple programs or devices, which do exactly one thing.
The software performs it's tasks one after the other, waiting as long as needed for the tasks to complete.}
\item{Cooperative multitasking - used for complex embedded applications in resource-constrained emvironments.
Timer tick is defined and upon each tick, the software performs all scheduled tasks.
The differences with the above are:}
\begin{enumerate}
\item{The tasks are performed once per specified time period, and not continuosly.
This is important for the PID algorithm.}
\item{As a consequence, tasks must be written with multitasking in mind.
Even a single process, which blocks for longer than the timer tick, would violate the real-time operation of the system.}
\end{enumerate}
\item{True preemptive multitasking - used for complex systems, such as personal computer operating systems.
This approach is far superior to all others, because programs are easily written, and it is the operating system's responsibility to schedule them efficiently.
Unfortunately, this is a complex taks, utilizing expensive (id est slow and memory consuming) algorithms.
Nevertheless, there exist lightweight multitasking operating systems for atmega devices.}
\end{itemize}
% TODO: quote a couple of references
The author has selected the secodnd approach.

\subsubsection{Implementation}
The file 'src/main.c' performs scheduling of the services, provided by the remaining modules.
The author has put effort into making all other modules independent of each other.
This is expected to provide the following benefits:
\begin{enumerate}
\item{The code is simple and easy to understand.}
\item{The implementation is not prone to error, due to chaning an unrelated portion of the code.}
\item{Software modules are easy to remove from the current project and insert into another, unrelated, project.}
\end{enumerate}

\subsection{Modules}
\subsubsection{Configuration}
Following best programming practices, the author has exported global project setting to the file 'src/config.h'.
From there it is easy to control:
\begin{itemize}
\item{all serial communication options}
\item{hardware bindings id est which pins fufill which tasks}
\item{thermometer library settings}
\item{is error checking to be performed}
\end{itemize}

\subsubsection{Clock}
This module utilizes a separate hardware timer to provide timekeeping and event management services.
The module keeps track of relative time, that is seconds elapsed from turn-on.
The clock wrap period is 136 years.
\par
Event management is acheaved through the function clock\_sleep\_until\_next\_second().
This is the essence of the multitasking system.

\subsubsection{Serial communication}
The built-in 'Universal sinchronous and asynchronous receive and transmit unit' or USART module is utilized in an elegant way.
%TODO: quote the datasheet
By providing implementation to functions put\_char() and get\_char(), the usart software modules actually binds 'stdin', 'stderr' and 'stdout' streams to the USART unit.
This fasciliates usage of standard stream operations, such as 'printf()' and 'scanf()'.
An added advantage is that the serial communiation implementation can be changed - e.g. to flow over USB or Bluetooth, and zero client code will need to be modified.

\subsubsection{Commands}
Utilizing the multitasking architecture of the software, a 'commands' module is implemented, which accepts user instructions at all possible time points, irrespective of currently executed tasks.
An added advantage of the commands system is that the device is fully scriptable.
Elaborating, a 'bash' script can issue commands to the system and read status back.
Such a script can, for example, log temperature data and performed control actions by the device.
As a matter of fact, such a script is provided, together with a Qt4 implementation of a temperature data visualization tool.
For more details, see the section Plotting tool

\subsubsection{Zero-cross detector}
An elaborate algorithm for mains voltage control was implemented.
The following options were considered:
\begin{itemize}
\item{Perform pulse width modulation on the control signal to the heater.
The sinusoidal nature of mains voltage introduces an absolute timing error of up to $2 * 10\si{\milli\second}$.
In the author's oppinion, with this simple scheme, only on-off control is possible.}
\item{Perform classical phase control.
This approach has proven itself over the years.
Furthermore, extensive literature is available, concerning this well-understood phenomenon.}
%TODO: quote Shishkov
\item{Measure AC phase and turn on only specific integer number of half-waves.}
\end{itemize}
As a PID control algorithm is desired, the first approach is immediately discarded.
The author moved away from the second approach for the following reasons.
The device in it's entirety is connected to the mains in a single point.
Phase control not only produces significant electro-magnetic interference, but also requires a zero-cross detector.
Such a detector relies on the clan waveform of the measured signal.
Consequenctly, in such a scheme, the triac would be injecting noise into the zero-cross sensor, nessetiating extensive filtering, with all of the associated complexity and latency.
\par
Consequencly, the author has selected the final approach.
By paying the price of less resolution than actual analog phase control, the following benefits have been attained:
\begin{itemize}
\item{Significantly reduced EMI - the selective activation of separate half-periods introduces a harmonic of $100\si{\hertz}$ frequency.
However, it's power is believed to be significantly lower than the combined power of the infinite spectrum ofa  phase-controlled system.}
\item{Simplified control - phase control would have nesseciated floating point numbers.
Linking in the floating point math library is expensive (both execution speed and storage space wise).}
\item{Indeterminate - the IEEE standard for floating point implementation in computers highlights numerous cases, where the result of a computation is implementation-dependent.
In contrast, an integer number implementation is always fully deterministic.}
%TODO: quote
\end{itemize}
\par
But wait!
The author has identified an algorithm to increase the precision to theoretically infinite value.
By utilizing a variation of Bresenham's line algorithm, comulative error is eliminated given enough time.
% TODO: quote
Furthermore, the algorithm supports safe modification of the setpoint, without relying on the fact, that the system tick is of known duration.
The implementation of the algorithm is quite simple and can be observed in file src/zcd.c, being contained in the function shoudl\_turn\_on();

\subsubsection{Triac control}
\subsubsection{Temperature measurement}

\subsection{Plotting tool}
Plotting a signal in real time can be acheaved by numerous approaches.
Among them are MATLAB, Scylab, Mathematica, Qt etc.
However, those that are not proprietery, are still badly suited to real-time operation.
Thus were selected the 'python' and 'matplotlib' technologies for their versability.
\par
The written plotting tool is called 'rtplot' - RealTime Plotting utility.
It operates on a vector of signals, plotting them against different time periods.
The longer time periods obtain their datapoints via algebraic averaging of the next shortest plot.
The program output covers the whole display and presents axis labels and units of measure.
Inidividual datapoints are also displayed, in addition to the interpolating line between them.
