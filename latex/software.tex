\section{Software}
The device is a typical realtime embedded system.
As such, mainframe programming tecniques are not applicable and embedded systems approach is required.
Consequently, the following requirements are layed out to the program:
\begin{itemize}
\item{Conveninet programming - during development, reprogramming the device shall be quick and easy, preferrably consisting of a single action.}
\item{Convenient communication - both debug data and process information shall be easily accessible during device operation.}
\item{Error tolerance - the device shall not freeze up upon an error, but should instead make best effort to keep the control loop active.}
\item{Determinism - the programmer needs to have full control over what is executed when.}
\item{Scriptability - the device shall be able to get reconfigured without a human at the other end id est a PC program shall be provided, which reconfigures the device in arbitrary ways.}
\end{itemize}
In the oppinion of the author, all of these requirements have been fully fufilled with the current version of the software.
The written software is free and open-source in it's entirety - it is protected under the MIT license.

\subsection{Bootloader}
\subsubsection{Motivation}
Several approaches were evaluated to reprogram the chip.
\par
As the chip resides in a socket, instead of being soldered to the Main board, it could be removed from it's socket and inserted into a programmer.
This approach is useful, but was discarded as extremely time-consuming.
\par
The next possible approach is In-system programming.
%TODO: quote something
This constitutes soldering a header onto the Main board, and connecting it to the SPI bus of the microcontroller.
This approach was also rejected for hte following reasons.
The author was relucant to pay both the price in real estate on the PCB and the aditional complexity of wirein the header to the microcontroller.
Not to mention using up some pins.
Last but not least, this approach would significantly complicate the RESET pin nooise protection circuit.
\par
The most complex, but in the same time best performing, approach was selected.
Utilizing the fact that serial communication to a PC is needed for other reasons, the author decided to transfer new programs over the same bus.
This way the device can be reprogrammed purely bu software, possibly even from a remote location.

\subsubsection{Previous work}
Many bootloader programs exist for the selected chip.
%TODO: quote 3-4
Unfortunately, all of them are either:
\begin{itemize}
\item{bloated with unnececery functionality, thus large in size,}
\item{do not work out of the box and thus require manipulation of the source code to work or}
\item{are licensed under un incompatible license than the veri permissive MIT license.}
\end{itemize}

\subsubsection{Implementation}
The bootloader was implemented in standard-conforming C language, and written by the author in it's entirety.
It is nearly impossible to create smaller in size bootloader for this device.
%TODO: quote avr-freaks
The program code is simple and straighforward, lacking complicated preprocessor directives.
Consequencly, it is in the oppinion of the author, easy to read and modify by anyone.
The code is of cource publically visible and licensed in a way, that any person can freely and legally modify and redistribute it.
\par
The bootloader instructions reside at the high end of the flash memory, while the application resides in the low end.
%TODO: quote the datasheet
Consequencly, the application never knows that a bootloader program was installed.
\par
The finished product was called 'megaboot' - a bootloader for atmega devices.
The source code in its entirety is provided as appendix A1.

\subsubsection{Communication protocol}
The XMODEM protocol has been selected.
%TODO: quote
This protocol is not used in modern high-bandwidth, high-complexity networking.
To the contrary. this protocol was popular when computers posessed resources comparable to the selected microcontroller.
It is a simple protocol with very little overhead (channel efficiency is 97\%).
Furthermore, it is easy to implement in few program instruction, using up small amount of flash memory.

\subsubsection{Error checking}
Erorr checking is provided throughtout the program:
\begin{itemize}
\item{CRC sums onto each received XMODEM packet}
\item{"magic character" starts each packet}
\item{addressing wrong (inexistient) flash pages is protected against}
\item{receiving packets out of order is also checked agains}
\end{itemize}
All the error checking code is conditionally included via the only preprocessor definition.
This serves twin purpose.
Not only does it clearly indicate error checking code apart from the actual program logic, but also provides the option to remove all error checking, should a person attempt to reduce the program size to the next smaller option.

\subsubsection{Use}
The build system passes the symbol BOOTLOAD to the application program.
This symbol holds the address of the first instruction of the bootloader.
The application is then free to jump to that address whenever required.
In an assembler program, this would have been performed via a RJMP instruction.
On the other hand, higher level languages cast the pointer to a function call address (optionally with the \_Noreturn attribute) and call it.
\par
After the bootloader has been invoked, it expects program data over the communication channel.
Fortunately, many applications are available, which support the XMODEM protocol.
One example is the popular and free serial terminal 'minicom'.

\subsection{Architecture}
