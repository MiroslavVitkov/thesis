\section{Hardware}
\subsection{Definitons}
The entirety of all physical components, resulting from this project, will be called 'device' throughout the paper.

\subsection{Goals}
The device is intended as a learning project for the student, but also as a open software, open-hardwre project, which anyone can create and use.
Thus, the following hardware design priorities have been identified, in order of decreasing importance:
% TODO: create a custom environment, which bolds the word before the dash
\begin{itemize}
\item{safety - the device shall not pose a fire or electric shock hazard to the end user}
\item{reconstructability - the device shall be composed \textbf{only} of worldwide accessible components}
\item{longivity - the device shall remain operational for 5 years of uninterrupted service with 95\% confidence}
\item{price - the BOM for the complete device shall not exceed 100BGN}
\item{extendability - the number of input sensors and the number of output controllers, shall be trivially configurable}
\item{ease of assembly - it shall be possible for a person with zero hardware experience to manufacture the device}
\item{simplicity - each component shall fufill a specific purpose, and the number of components shall be the lowest possible}
\end{itemize}

\subsection{Layout}
Due to the requirement of extendability, the device shall consist of a number of printed circuit boards, in contrast to a single  monoliotic PCB.
Each PCB shall fufill a sole purpose, and any number of different modules shall be able to mate together.
The following distinct roles have been identified:
\begin{itemize}
\item{high-voltage input stage - called zero-cross detector or ZCD board from now on}
\item{low-voltage input stage - called temperature sensor or thermometer from now on}
\item{computational stage - called main board from now on}
\item{high-voltage output stage - called software controlled rectifier board or SCR board from now on}
\end{itemize}
The resulting design exhibits the following characteristics.
\par
Only a single ZCD board is required, because mains waveform is invariant across the device in it's entirety.
Only a single main board is required, as the selected microcontroller, although inexpensive, provides plenty of resources for numerous control loops.
\par
In order to satisfy the requirement for simplicity, the main board is configured for a single SCR output board.
However, soldering aditional connectors to the main PCB is trivial, thus acheaving extensability.
The SCR output board is long-life and supports loads of up to 1\si{\kilo\watt}.
\par
The most flexible part of the system is the thermometer configuration.
Due to the selected temperature sensing IC, virtually unlimited (technically up to $2^{56}$) devices are supported \textbf{without any hardware changes}.
% TODO: on the aboce, quote http://datasheets.maximintegrated.com/en/ds/DS18S20.pdf

\subsection{ZCD board}
The ZCD board is a sensory input to the microcontroller.
\par
Because the voltage of mains power is alternating, it is impossible to output precise amounts of power without knowing the phase of the waveform.
The implementation of the ZCD board is straightforward and extremely simplified.
In fact, an extensive internet search has demonstrated no other PCB has ever been designed with such a level of simplicity.
In other words, \textbf{the designed PCB contains fewer elements than any known PCB for the same purpose}!
This produces problems, which have deterred other designers.
However, all artifats have been dealt with in software.

\subsubsection{Schematic}
Please refer to appendix A1 for the schematic and layout of the board.

\subsubsection{Calculation}
In order to protect the main board (and thus the user) from dangerous voltages, galvanic isolation is required.
The standard means to this end are transformers and optocouplers.
Optocouplers posess numerous advantages over transformers for our application:
\begin{itemize}
\item[--]{compactness}
\item[--]{low price}
\item[--]{neglegable phase shift}
\end{itemize}
Furthermore, among optocouplers, the variation is considerable.
We select a component with anti-parallel input LEDs, specifically designed for zero crossing - SFH620A-3.
This is the most sensitive version of the IC (highest CTR), as input power is our greates concern.
\par
Striving for minimal component count and price, the standard solution with a 10W input power resistor is dismissed.
Thus, we need to work with 1/4W, E24 resistors.
Due to the optocoupler's acceptable CTR, and extensive signal conditioning in software, this solution will provide to be viable!
\par
Let's suppose the line voltage varies from $V_{min} = 200VAC$ to $V_{max} = 250VAC$ rms.
$$ P_{input resistors} = \frac{V_{max}^2}{R_1 + R_2}$$
$$ R_1 + R_2 \geq \frac{V_{max}^2}{P_{input resistors, max}} = \frac{250^2}{0.25+0.25} = 125\si{\kilo\ohm}$$

We select R1 = R2 = 68Kohm.
$$ i_{in, min} = \frac{V_{min} - 1.65}{2 * R_1 *1.05} = \frac{198.35V}{142.8Kohm} =  1.39\si{\milli\ampere}$$

%From here on, using [these](http://electronics.stackexchange.com/a/33043/9910) or [these](http://electronics.stackexchange.com/a/93597/9910) calculations, we select 10Kohm or 15kohm for the output resistor.

%Conclusion: minimum component count and price, borderline reliability, should work for a couple of years at least.

%*Would the schematic work? Would it be reliable for a couple of years of constant operation?*

%---

%The output stage:
%
%$$  i_C \geq i_{in, min} * CTR_{min} \approx 1.39 mA * 0.34 \approx 0.47mA $$
%
%[atmega168](http://www.atmel.com/images/atmel-42176-atmega48pb-88pb-168pb_datasheet.pdf), page 302
%
%$$ i_{leakage} \leq 1uA $$
%$$ V_{IL} = 0.3 Vcc = 0.3 * 5V = 1.5V $$
%$$ V_{IH} = 0.6 Vcc = 0.3 * 5V = 3V $$
%
%If we strive to be below 1V for logic zero:
%$$ i_C * R_{output} = 1V $$
%$$ R_{output} = 1V / 0.47mA = 2.13Kohm$$
%
%We select 2.4Kohm.
